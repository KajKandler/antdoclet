\documentclass[letterpaper,11pt,oneside]{article}

%% This document servers as a Template and introduction to your Ant
%% Tasks. At the end, it includes the output of AntDoclet, which will
%% provide all the reference documentation for your Ant tasks.


\usepackage{apache}
\usepackage{relsize} % \smaller and \larger commands (to change fontsize relative to the current size).

% Important: The nameref, varioref and hyperref packages must be
% loaded in that order for proper functioning.
%
\usepackage{nameref} % Nameref (part of Hyperref package) for referencing sections by name (instead of number)
\usepackage{varioref} % for 'smart' references
\usepackage{calc}     % allow for algebraic calculations in length arguments
\usepackage{draftcopy}
\draftcopyName{Example}{200}

\lstset{escapechar=`}
\sloppy % allow for not-so-nice linebreaks
\usepackage[htt]{hyphenat} % Allow for TypeWriter text (ie \texttt{...}) to be hyphenated

\newcommand{\webLink}[2]{\href{#2}{#1}\footnote{\href{#2}{#2}}}




% specify the title/author
\title{AntDoclet Example}
\author{Fernando Dobladez\\\smaller\slshape{}fernando@dobladez.com}

% set the PDF Document Info
\hypersetup{
  pdftitle    = {Apache Ant Integration},
  pdfsubject  = {ANT},
  pdfkeywords = {ant,script,automation},
  pdfauthor   = {Fernando Dobladez},
}


%%%%%%%%%%%%%%%%%%%%%%%%%%%%%%%%%%%%%%%%%%%%%%%%%%%%%%%%%%%%%%%%%%%%%%%%%%%%%%
\begin{document}
\maketitle % generates the title page
\newpage
\begin{abstract}
This document serves as reference documentation for the core Apache Ant tasks.

This is NOT the official \webLink{Apache Ant}{http://ant.apache.org}
documentation. It is exposed here just as an example of what
\webLink{AntDoclet}{http://antdoclet.neuroning.com} can generate.

\end{abstract}

\tableofcontents
\newpage


%%%
\section{Introduction}
This documentation was generated running
\webLink{AntDoclet}{http://antdoclet.neuroning.com} over the source
code of the core Apache Ant Tasks version 1.6.5.

This NOT the official \webLink{Apache Ant}{http://ant.apache.org}
documentation. It is exposed here just as an example of what
\webLink{AntDoclet}{http://antdoclet.neuroning.com} can generate.


\subsection{About this example}
The source code of the core Ant Tasks was (obviously) written without
any knowledge of AntDoclet. Therefore, the quality of the generated
documentation for this example is far from perfect.

For instance, the source code of the core Ant Tasks do not use all the
tags that AntDoclet supports, and some of the comments use invalid
HTML or describe things that AntDoclet already auto-generates.

To get the most out of AntDoclet the source code of your Ant Tasks
needs to follow a few simple guidelines described in the AntDoclet
documentation.

\vspace{1cm}
\hrule
The following sections describe each of the provided Ant Tasks in
more detail, including all the attributes they accept and examples of
use.


%%
%% Include the output of AntDoclet (Ant Tasks reference documentation)
%%
\include{anttasks-reference}

\end{document}
